% cv-us.tex
% $Id: cv-us.tex,v 1.32 2007/11/06 13:59:29 jrblevin Exp $
%
% LaTeX Curriculum Vitae Template
%
% Copyright (C) 2004-2007 Jason Blevins <jrblevin@sdf.lonestar.org>
% http://jrblevin.freeshell.org
%
% You may use use this document as a template to create your own CV
% and you may redistribute the source code freely. No attribution is
% required in any resulting documents. I do ask that you please leave
% this notice and the above URL in the source code if you choose to
% redistribute this file.
%
% Durham, February 17, 2007
%
%%---------------------------------------------------------------------------%
%
% Notes:
%
% * THIS IS IMPORTANT.
%   Don't forget to change `pdfauthor' and `keywords' in the \hypersetup
%   section below. Lots of other people's CVs show up in Google when I type my
%   name because they didn't change these and my name is in their PDF metadata.
%
% * To create a new page use: \newpage \opening
%
% * res.cls includes an \address{} command which can be used up to twice,
%   but my address is too long for the format it uses.
%
% * Alternate documentclass statement to put headings in margin:
%   \documentclass[margin,line,11pt,final]{res}
%
% * Can divide publication/presentation list into subsections by year:
%   \section{\bf\small\hspace{8mm}2006}
%
%%----------------------------------------------------------------------------%

\documentclass[overlapped,line,letterpaper]{res}

\usepackage{ifpdf}

\ifpdf
  \usepackage[pdftex]{hyperref}
\else
  \usepackage[hypertex]{hyperref}
\fi

\pagestyle{plain}

\font\ss=cmss10

\hypersetup{
  letterpaper,
  colorlinks,
  urlcolor=black,
  pdfpagemode=none,
  pdftitle={Curriculum Vitae},
  pdfauthor={Reza Akhtar},
  pdfcreator={$ $Id: cv-us.tex,v 1.32 2007/11/06 13:59:29 jrblevin Exp $ $},
  pdfsubject={Curriculum Vitae},
  pdfkeywords={mathematics algebraic geometry combinatorics}
}

%%===========================================================================%%

\begin{document}

%---------------------------------------------------------------------------
% Document Specific Customizations

% Make lists without bullets and with no indentation
\setlength{\leftmargini}{0em}
\renewcommand{\labelitemi}{}

% Use large bold font for printed name at top of pages
\renewcommand{\namefont}{\large\textbf}

% Make a nicer C++
\def\Cplusplus{C{\raise.5ex\hbox{\footnotesize ++ }}}

%---------------------------------------------------------------------------

\name{Reza Akhtar, Ph.D.}

\begin{resume}

\begin{ncolumn}{2}
  Department of Mathematics  & Phone: (513) 529-1902 \\
  Miami University                    & Fax: (513) 529-1493 \\
  Oxford, OH 45056               & E-mail: {\tt akhtarr@miamioh.edu}  \\
                                              & URL: {\tt 
\verb+www.users.miamioh.edu/akhtarr+} \\
\end{ncolumn}

%---------------------------------------------------------------------------

\large
\section{\bf Education}
\vspace{3 mm}
\normalsize

\textbf{1995} ~ ~ A.B. Mathematics {\em magna cum laude}, Harvard University \\
$ ~ ~ ~ ~ ~ ~ ~ ~$ ~ ~ {\ss Thesis}: {\em Cyclotomic Euclidean Number Fields} \\
$ ~ ~ ~ ~ ~ ~ ~ ~ $ ~ ~ {\ss Thesis advisor}: Barry Mazur. 

\textbf{1995} ~ ~ S.M. Applied Mathematics, Harvard University.

\textbf{1997} ~ ~ Sc.M. Mathematics, Brown University.

\textbf{2000}  ~  ~ Ph.D. Mathematics, Brown University \\
$~ ~ ~ ~ ~ ~ ~ ~$ ~ ~ {\ss Dissertation}: {\em Milnor $K$-theory and zero-cycles on algebraic varieties} \\
$ ~ ~ ~ ~ ~ ~ ~ ~$ ~ ~ {\ss Dissertation advisor}: Stephen Lichtenbaum \\
$ ~ ~ ~ ~ ~ ~ ~ ~ $ ~ ~ {\ss Areas of emphasis}: Algebraic geometry, algebraic cycles. 

\textbf{2021} ~ ~ Graduate course work in Philosophy and Historical Theology, Catholic Distance University. \\


\large
\section{\bf Language Proficiency}
\vspace{3mm}
\normalsize

Native: English \\
Professional working knowledge: French \\
Working knowledge: Latin, Arabic, Biblical Hebrew, Syriac, Koine Greek, Urdu, German


%---------------------------------------------------------------------------

\large
\section{\bf Employment}
\vspace{3 mm}
\normalsize
\textbf{July 2012 - present     } $ ~ ~  ~ ~ ~ ~ ~ ~ ~$ Professor, Department of Mathematics, Miami University. \\
$~ ~ ~ ~ ~ ~ ~ ~ ~ ~ ~  ~ ~ ~ ~  ~ ~ ~ ~  ~ ~ ~ ~ ~ ~ ~ ~ ~~ ~ ~  ~  ~ ~ ~ ~ ~~ ~ ~ ~  $ {\ss Permanent Graduate Level A status, December 2018.} \\
\textbf{July 2006 - June 2012} $ ~ ~ ~ ~ ~ ~ ~ $ Associate Professor, Department of Mathematics, Miami University. \\
\textbf{August 2000 - June 2006} $ ~ ~ $ Assistant Professor, Department of Mathematics, Miami University. \\



%---------------------------------------------------------------------------



\large 
\section{\bf Research Publications}
\vspace{3 mm} 
\normalsize

{\ss My dissertation and early papers are in 
algebraic geometry, specifically algebraic cycles.  Most of the 
rest of my work is concerned with questions at the intersection of algebra and 
combinatorics.}

\smallskip
R. Akhtar and S. Gagola III. Strong complete mappings for products of $2$-groups. In preparation.

\smallskip
*R. Akhtar and S. R. Arvind.  On the distribution of the greatest common divisor for number fields.  In preparation.

\smallskip
R. Akhtar. Linear operator identities in quasigroups. To appear in {\em Comm. Math. Univ. Carolinae}.

\smallskip
R. Akhtar and S. Gagola III.  Strong complete mappings for $3$-groups.  {\em Discrete Math.} \textbf{21} (2022), Paper no. 112643.

\smallskip
+R. Akhtar and R. Joshua.  Explicit Chow-Lefschetz decompositions of Kummer manifolds.  $K$-theory: Proc. of the International Colloquium, Mumbai, 2016.  Hindustan Book Agency (2018), 155-194.  

\smallskip
R. Akhtar.  Symmetric linear operator identities in quaisgroups.  {\em Comm. Math. Univ. Carolinae} \textbf{58} (2017), no. 4, 401-417.

\smallskip
R. Akhtar.  On generalized associativity in groupoids.  {\em Quasigroups and Rel. Sys.} \textbf{24} (2016), 1-6.

%R. Akhtar, {\em Kato-Somekawa $K$-groups over a discrete valuation ring}, in 
%progress.  
\smallskip
*R. Akhtar and L. Lee.  Connectivity of the zero-divisor graph for finite rings.  {\em Involve } \textbf{9} (2016), no. 3, 415-422.

\smallskip
*R. Akhtar, B. Burns, H. Hoganson, H. Mansfield, O. Sobieska, and Z. Woods.  
Splitting techniques and the Betti numbers of secant powers. {\em Involve} \textbf{9} (2016), no. 5, 737-750.


\smallskip
*R. Akhtar and M. Forlini.  The Linear Chromatic Number of a Sperner 
Family. {\em Discrete Applied Mathematics} \textbf{171} (2014), 1-8.  


\smallskip
R. Akhtar, A. B. Evans, and D. Pritikin.  Representation Numbers of Complete 
Multipartite Graphs.  {\em Discrete Mathematics} \textbf{3112} (2012), 1158-1165.

\smallskip
*R. Akhtar, A. Arp, M. Kaminski, J. VanExel, D. Vernon, and C. Washington. 
The varieties of Bol-Moufang quasigroups defined by a single operation.
{\em Quasigroups and Rel. Sys.} \textbf{20} (2012), 1-10.

\smallskip
R. Akhtar.  Representation Numbers of Some Sparse Graphs. {\em Discrete 
Mathematics} \textbf{312} (2012), no. 22, 3417-3423.

\smallskip
R. Akhtar and R. Joshua.  Toric Residue Codes: I.  {\em Finite Fields and 
their Applications} \textbf{17} (2011), no. 1, 15-50.    


\smallskip
R. Akhtar and P. Larson.  Small-sum pairs in abelian groups.  {\em J. de 
Th. des Nombres de Bordeaux} \textbf{22} (2010), no.3, 525-535.

\smallskip
R. Akhtar, A. B. Evans, and D. Pritikin.  Representation Numbers of Stars.  
 {\em Integers} \textbf{10} (2010), 733-745.


\smallskip
*R. Akhtar, M. Boggess, T. Jackson-Henderson, I. Jim\'{e}nez, R. Karpman, A. 
Kinzel, and D. Pritikin.  On the unitary Cayley graph of a finite ring.  
{\em Elec. J. of Combinatorics} \textbf{16} (2009), no. 1, Research Paper 117, 
13 pp.


\smallskip
R. Akhtar and R. Joshua.   Lefschetz Decompositions for Quotient Varieties. 
  {\em Journal of }$K${\em -theory} \textbf{3} (2009), no.3, 547-560.

\smallskip
+R. Akhtar, P. Brosnan, and R. Joshua, eds. {\em The Geometry of Algebraic Cycles}. 
Papers from the 2nd Conference on Algebraic Cycles held at the Ohio State 
University, Columbus, OH.   March 25–29, 2008. Clay Mathematics Proceedings 
\textbf{9}.


\smallskip
R. Akhtar, T. Jiang, and Z. Miller.  Asymptotic determination of 
edge-bandwidth of multidimensional grids and Hamming graphs.   
{\em SIAM J. on Discrete Mathematics} \textbf{22} (2008), no. 2., 425-449.

\smallskip
R. Akhtar, T. Jiang, and D. Pritikin.  Edge-bandwidth of the triangular 
grid. {\em Elec. J. of Combinatorics} \textbf{14} (2007), no. 1, Research Paper 67, 11 pp.

\smallskip
R. Akhtar.  A mod-$\ell$ vanishing theorem of Beilinson-Soul\'{e} 
type.  {\em J. of Pure and Applied Algebra} \textbf{208} (2007), no. 2, 555-560.

\smallskip
*R. Akhtar and L. Lee.  Homology of zero-divisors.  {\em Rocky Mountain 
J. of Mathematics} \textbf{37} (2007), no. 4, 1105-1126.

\smallskip
R. Akhtar and R. Joshua.  K\"{u}nneth decompositions for quotient 
varieties.  {\em Indagationes Mathematicae} \textbf{17} (2006), no. 3, 319-344.

\smallskip
R. Akhtar.  Cycles on curves over global fields of positive 
characteristic.  {\em Trans. of the Amer. Math. Soc.} \textbf{357} (2005), 2557-2569.     

\smallskip
R. Akhtar.  Adequate equivalence relations and Pontryagin products.  {\em 
J. of Pure and Applied Algebra} \textbf{196} (2005), no. 1, 21-37.     

\smallskip
R. Akhtar.  Milnor $K$-theory of smooth varieties. 
 {\em $K$-theory} \textbf{32} (2004), no. 3, 269-291.

\smallskip
R. Akhtar.  Torsion in mixed $K$-groups. 
{\em Communications in Algebra} \textbf{32} (2004), no. 1, 295-313.

\smallskip
R. Akhtar.  Zero-cycles on varieties over finite fields. 
{\em Communications in Algebra} \textbf{32} (2004), no. 1, 279-294. 

\smallskip
R. Akhtar and A. Lachlan.  On countable homogeneous 3-graphs. 
{\em  Archive for Mathematical Logic}  \textbf{34} (1995), no. 5, 331-344.

\medskip

* = joint work with students, + = conference volume 

\large
\section{\bf Research Presentations}
\vspace{3 mm}
\normalsize

\smallskip
Quasigroups, generalized associativity, and automated theorem-proving. \\ {\ss Colloquium, Wright State University, November 2017.}

Betti numbers of secant powers of the edge ideal of a graph.  \\ {\ss MIGHTY LVII, Wright State University, April 2016.}

Explicit motivic decompositions for Kummer varieties and manifolds. \\ {\ss Seminar, Ohio State University, November 2015.}

Splitting techniques and the Betti numbers of secant ideals. \\ {\ss Colloquium, Ohio State University, April 2014.}

Representation numbers of complete multipartite graphs.\\ {\ss MIGHTY LII, Indiana State University, April 2012.}

Bol-Moufang quasigroups defined by a single operation. \\ {\ss SIDIM, University of Puerto Rico - Humacao, February 2011.}

Small-sum pairs in abelian groups.  \\ {\ss CMS Winter Meeting, Vancouver, BC.  December 2010.}

The Linear Chromatic Number of a Sperner Family. \\ {\ss MIGHTY L, University of Wisconsin -- Superior,  October 2010.}

The zero-divisor graph: at the intersection of algebra and
combinatorics. \\ {\ss Colloquium, Butler University, October 2009.}

The Beilinson-Soul\'{e} Conjecture with finite coefficients. \\ {\ss Algebraic 
Cycles Conference II, Ohio State University, March 2008.}  

Motivic decompositions for quotient varieties. \\ {\ss Algebraic
Geometry Seminar, Ohio State University, March 2007.}

Chow-K\"{u}nneth and Lefschetz Decompositions for Quotient
Varieties".  \\ {\ss CMS Winter Meeting, Toronto, ON.  December, 2006.}

Elliptic Curves, Arithmetic, and Geometry. \\ {\ss Colloquium,
Baldwin-Wallace College, November 2006.}

The zero-divisor graph: at the intersection of algebra and
combinatorics.  \\ {\ss Undergraduate Seminar, Trinity University, September
2006.}

Combinatorial methods for studying zero-divisors. \\ {\ss Colloquium,
Trinity University, September 2006.}

Beyond the zero-divisor graph: a homology theory for
zero-divisors. \\ {\ss Colloquium, Wabash College, June 2006.}

Edge-bandwidth of the triangular grid. \\ {\ss MIGHTY XLII: Ohio State U. -- Marion Campus, April 2006.}

A vanishing theorem of Beilinson-Soul\'{e} type.  \\ {\ss $K$-theory
Seminar, Ohio State University, March 2006.}

Cycles, cohomology, and motives. \\ {\ss Colloquium, Miami University, April 
2005.}

Algebraic cycles on curves over global fields.  \\ {\ss Algebraic
Geometry Seminar, Ohio State University, April 2005.}

Algebraic Cycles on Abelian Varieties.  \\ {\ss Colloquium, Rose-Hulman Institute 
of Technology, November 2004.}

Elliptic Curves, Arithmetic and Geometry.  \\ {\ss Colloquium, Wabash College, 
March 2004.}

Cycle groups of curves over global fields of positive characteristic.  \\ {\ss Joint Mathematics Meetings.  Phoenix, AZ, January 2004.}

\smallskip
Cycles on Algebraic Varieties.  \\ {\ss Colloquium, University of Dayton, October 2002.}

\smallskip
Adequate equivalence relations and cycles on abelian varieties.  \\
{\ss AMS Regional Meeting, Boston University, October 2002.}

\smallskip
Kato-Somekawa groups and higher Chow groups of zero-cycles. \\ {\ss AMS Regional Meeting, Ohio State University, September 2001.}

\smallskip
Milnor $K$-theory of smooth schemes.  \\ {\ss AMS Regional Meeting, U. of Kansas, April 2001.}

\smallskip
Milnor $K$-theory and Intersection Theory.  \\ {\ss Colloquium, University of 
Cincinnati, January 2001.}

\smallskip
Zero-cycles on algebraic varieties.  \\ {\ss Algebra Seminar, University of Pennsylvania, January 2000.}


\large
\section{\bf Teaching and Advising}
\vspace{3 mm}
\normalsize

{\bf At Brown University}

\begin{itemize}

\item
MTH 9, Calculus I: {\ss Summer 1998}.

\item
MTH 17, A.P. Calculus II: {\ss Fall 1998.}


\end{itemize}


{\bf At Miami University}

\begin{itemize}

\item
MTH 151, Calculus I: {\ss Fall 2000, Fall 2001, Fall 2005.}

\item
MTH 153, Calculus I: {\ss Fall 2009.}

\item
MTH 190, First Year Seminar in Mathematics and Statistics: {\ss Spring 2013, 
Fall 2013, Fall 2014.}

\item
MTH 222, Linear Algebra: {\ss Fall 2000, Fall 2007 (2 sec.), Spring 2013, Fall 
2013, Fall 2014, Spring 2016, Spring 2017 (2 sec.), Spring 2022 (2 sec.)}

\item
MTH 231, Discrete Mathematics: {\ss Spring 2003, Spring 2005 (2 sec).}

\item
MTH 245, Differential Equations for Engineers: {\ss Spring 2010, Fall 2010, Fall 
2011, Fall 2012, Fall 2015, Fall 2017, Spring 2018, Fall 2019, Spring 2020, Spring 2021 (2 sec.).}

\item
MTH 247, Financial Mathematics for Actuaries.  {\ss Spring 2015, Spring 2016.}


\item
MTH 249, A.P. Calculus II: {\ss Fall 2002, Fall 2003, Fall 2006, Fall 2013, Fall 2020 (2 sec.).}

\item
MTH 249H Honors A.P. Calculus II: {\ss Fall 2007, Fall 2012.}

\item
MTH 251, Calculus II: {\ss Spring 2001, Spring 2006, Fall 2008, Spring 2011, Summer 2017 (first third), Fall 2019.}

\item
MTH 252, Calculus III: {\ss Fall 2004, Spring 2007, Spring 2012, Spring 2014, Fall 2017, Fall 2018, Summer 2021, Fall 2021.}

\item
MTH 252H, Honors Calculus III: {\ss Spring 2008.}

\item
MTH 347, Differential Equations: {\ss Fall 2006.}

\item
MTH 420/520, Topics in Algebra: {\ss Spring 2004, Summer 2016.}

\item
MTH 421/521, Abstract Algebra I: {\ss Fall 2001, Spring 2003, Fall 2005, Spring
2007, Fall 2009, Fall 2011, Spring 2012, Fall 2014, Fall 2016, Fall 2018, Spring 2020.}

\item
MTH 422/522, Abstract Algebra II: {\ss Spring 2001, Spring 2010.}

\item
MTH 425/525, Number Theory: {\ss Fall 2003, Fall 2004, Spring 2014, Spring 2018.}

\item

MTH 447/547, Topics in Mathematical Finance: {\ss Spring 2011, Spring 2013, Fall 2016, Summer 2018, Summer 2020.}

\item
MTH 620, Topics in Algebra: {\ss Summer 2003.}

\item
MTH 621, Graduate Algebra I: {\ss Fall 2002, Fall 2008, Fall 2010, Fall 2015, Fall 2021.}

\item
MTH 622, Graduate Algebra II: {\ss Spring 2002, Spring 2006, Spring 2019.}

\end{itemize}


{\bf Undergraduate advising} \\
Since 2003, I have served as academic advisor for between five and twelve undergraduate students each year studying towards the B.S. in Mathematics.  I have completed the first four modules of the Advisor Training program and expect to be formally awarded Level B Advisor status within the next few weeks.  

{\bf Course development} \\
In 2015, I developed MTH 247 (Financial Mathematics for Actuaries) to provide students with preparation for the second exam in the Society of Actuaries sequence.  


\large
\section{\bf Supervised Student Research}
\normalsize

\vspace{3 mm}

{\bf Master's Theses} \\ 
M.A. Thesis advisor for Jeffrey Cooper, August 2009 - April 2010. \\ 
{\ss Thesis:} Product dimension of a random graph.

M.A. Thesis advisor for Daniel Baczkowski, August 2003 - July 2004. \\
{\ss Thesis: } Diophantine equations involving arithmetic 
functions of factorials.


%{\bf Master's final projects}

{\bf Master's final projects (M.A. or M.S. Mathematics)}  \\
\begin{itemize}
\item
Jacob Charboneau (currently in progress)

\item
Robyn Campbell (October 2018 - July 2019)

\item
Emmanuel Tamakloe (January 2015 - October 2015)

\item
Christine Stoller (January 2013 - August 2013)

\item
Joshua Fitzgerald (January 2012 - August 2012)

\item
Laura Hoffman (January 2011 - October 2011)

\item
Cory Washington (January 2010 - January 2013)

\item
Carmen Weddell (August 2009 - April 2010)

\item
Joshua Wagner (August 2008 - June 2009)

\item
Benjamin Byer (August 2006 - May 2007, project not completed)

\item
Holly Attenborough (August 2005 - May 2006)

\item
Melody Brickel (January - May 2004)

\item
Deborah Puffer (January - May 2003)

\item
Amy Herron (March - July 2002)


\end{itemize}

{\bf Graduate Independent Studies}

\begin{itemize}


\item
Jacob Barahona-Kaamsvag, Anthony Wilkie, Michael Woode, Ruifeng Xu, Summer 2019.

\item
Delaney Aydel, Summer 2017.

\item
Robert Seiver, Fall 2009.

\end{itemize}



{\bf Graduate Examinations}

\begin{itemize}

\item
Algebra Comprehensive Exam Committee (17 times since 2000).
% 1/01, 5/01, 1/02, 5/02, 1/03, 5/03, 1/04, 5/06, 1/09, 6/09, 9/09, 2/10, 1/11, 5/11, 1/16, 2/17, 5/19.

\item
Master's Final Exam Committee (2 thesis advisor, 12 final project advisor, 13 final project committee member)

%Herron, Baczkowski, Puffer, Brickel, Attenborough, Cooper, Wagner, Washington, Weddell, Hoffman, Fitzgerald, Stoller, Tamakloe, Campbell

%Others are Aukerman 2004, Ghattas 2006, Emmons, Seiver, Perkins, Miskimen 2010, McNamara, Craft (thesis) 2011, Irwin 2015, Aydel 2018, Melton, Searcy 2019 Barahona 2021
\end{itemize}

{\bf  Undergraduate Honors Thesis Reader}\\
Iordan Ganev, Spring 2010.





{\bf Undergraduate (Miami) research students}

\begin{itemize}
\item
S. Ram Arvind, Summer 2017.

\item
Maxwell Forlini, Summer 2009.

\item
Nathan St. John, Summer 2007.

\item
Lucas Lee, Summer 2003.
\end{itemize}

\newpage

{\bf Undergraduate SUMSRI research students}

\begin{itemize}


\item
2013: Brittany Burns, Haley Mansfield, Ola Sobieska, Zerotti Woods.

\item
2012: Rachel Aldrich, Sarah Drummond, Barbara Hernandez, Hannah Hoganson, Lauren 
Morey, Marco Tapia-Guilliams, Alicia Velek.

\item
2011: Crystal Altamirano, Stephanie Angus, Lauren Brown, Laura Gioco, Joseph 
Crawford.

\item
2010: Ashley Arp, Michael Kaminski, Jasmine Van Exel, Davian Vernon.

\item
2009: Daniel Caproni, Joshua Edgerton, Margaret Rahmoeller, Mychael 
Sanchez, Anna Tracy. 

\item
2008: Megan Bernstein, Megan Boggess, Tiffany Jackson-Henderson, Isidora Jim\'{e}nez, Rachel Karpman.

\item
2007:  Katherine Benson, Louis Cruz, Yesenia Cruz, Melissa Tolley, Bryant Watkins.

\item
2006: Chantelle Bicket, Samantha Graffeo, Darragh Ross, Edward Washington.

\item
2005: Camil Aponte, Natalia C\'{o}rdova, Clyde Gholston, Helen Hauser, Patrice Johnson, Nathan Mims.

\item 
2004: Amanda Phillips, Julie Rogers, Kevin Tolliver, Frannie Worek.

\end{itemize}

{\bf  Undergraduate Independent Studies}

\begin{itemize}
\item
Dylan Palo, Spring 2017.

\item
Kara Ungerman, Fall 2013.

\item
Jonathon Hall, Spring 2008.

\item
Todd Van Woerkom, Fall 2007.
\end{itemize}


\large
\section{\bf Awards}
\normalsize
\vspace{3 mm}

{\bf External funding} \\
Co-principal investgator on NSF grant ($\$250,000$) to support SUMSRI 2019 and 2020 (not awarded).

Co-principal investigator on NSF grant ($\$28,900$), travel for SUMSRI 2018 and 2019 (not awarded).

Co-principal investigator on NSA grant ($\$125,000$) to support SUMSRI 2018 (awarded).

Co-principal investigator on NSA grant ($\$118,804$) to support SUMSRI 2015 (awarded).

Co-principal investigator on NSF grant ($\$90,000$) to support SUMSRI 2015 (not awarded)

Co-principal investigator on NSA grant ($\$125,000$) to support SUMSRI 2014 (not awarded)

Co-principal investigator on NSF Grant  ($\$175,416$) to support SUMSRI 2013 and 2014 (awarded).

Co-principal investigator on NSA Grant  ($\$150,000$) to support SUMSRI 2013 (awarded).

Co-principal investigator on NSF Grant ($\$142,541$) to support SUMSRI 2011 and 2012 (awarded).

Co-principal investigator on NSA Grant,  ($\$431,270$) to support SUMSRI 2011 and 2012 (awarded).

Co-principal investigator on NSA Grant,  ($\$188,441$) to support SUMSRI 2010 (awarded).

%Received \$500 to attend Lichtenbaum birthday conference, March 2005.

%Received \$750 to attend Friedlander birthday conference, Sept. 2004.

%Received \$170 to attend Great Lakes $K$-theory conference, May 2004.

\newpage

{\bf Internal Funding}\\
$ $
Miami University USS (Undergraduate Summer Scholars) grants for 2003, 2007, 2009, 2017 to supervise student research in algebra and combinatorics.

Miami University College of Arts and Sciences Summer Research Grant (\$4000), 2000.

Miami University Committee for Faculty Research Summer Grant (\$6000), 2000.


{\bf Other awards and recognition}

\smallskip
\begin{itemize} 

\item
Student Recognition of Teaching Excellence Award, Fall 2020.

\item
\underline{M. Pauline Priest Barney fellowship}, 2016-2017. \\
{\ss \indent This fellowship is given to a faculty member in the Department of Mathematics for the specific purpose of developing a new course or redesigning an existing mathematics course.  In Fall 2014, I was asked to develop a course to prepare Actuarial Science minors for the second exam (Financial Mathematics) in the sequence for professional licensure.  I taught this course (MTH 247) for the first time in Spring 2015.  I then used the fellowship to study what might be improved in future offerings of the course.  I taught the course again in Spring 2016.  Unfortunately, in Spring 2017 and Spring 2018, the class had to be canceled due to low enrollment; student demand was assessed to be too low to warrant even putting it on the schedule for Spring 2019 or 2020.}

\item
Nominated for Alumni Distinguished Educator Award, 2008.

\item
\underline{Exxon-Mobil Project NExT Fellow}, 2001. \\
{\ss Project NExT is a professional development program for early-career mathematicians which has been administered by the Mathematical Association of America since 1994.  Fellows attend three national conferences, at which sessions are held to discuss and explore various issues of concern to new faculty members.  Project NExT also maintains a network of mentors and several mailing lists for further discussion and dissemination of information.}  

\end{itemize}

\large
\section{\bf SUMSRI Program}
\vspace {3 mm}
\normalsize

I was heavily involved in the Summer Undergraduate Mathematical Sciences Research Institute (SUMSRI), from early in my career (2002) until discontinuation of the program in 2018.  SUMSRI was a seven-week long program, hosted by the Department of Mathematics at Miami University, whose goals was to encourage talented undergraduates -- particularly those from underrepresented demographic groups -- to pursue research and graduate education in the mathematical sciences.  Students recruited from universities across the country were given the opportunity to conduct research in a seminar under the direction of a faculty member in mathematics or statistics; each research seminar was assisted by a graduate student in the mathematical sciences who also served as a mentor to the undergraduates.  SUMSRI also offered a sequence of short courses, a colloquium series, and a graduate panel discussion featuring representatives from programs at universities in the general area.  Until 2014, students were also given funding to attend the annual Joint Mathematics Meetings the following January to present their research in poster form.  

\medskip

\underline{Program Director}: (2014, 2018) \\
{\ss I wrote the grant proposals, advertised the program, recruited students, and selected between eight and seventeen students (dependent on the level of funding) from an applicant pool of roughly 200 students each year.  I made offers and assigned students to research seminars.  I was also responsible for setting the program schedule, inviting colloquium speakers, and organizing other program events.}

\underline{Program Director and Coordinator}: (2015) \\
{\ss As for Program Director, with the additional responsibility of organizing travel for students, processing paychecks, and keeping accounts of all program expenses.}

\newpage
\underline{Program co-Director}: (2010 - 2013) \\
{\ss As for Program Director, except that duties were shared with Program co-Director Patrick Dowling (Department of Mathematics).}

\underline{Research Seminar Director}: (2003 - 2013) \\
{\ss During the seven weeks of the program, I met with an assigned group of four to seven students and supervised their 
research on a problem of my choosing.  The students wrote up their results in a final paper and delivered a final presentation.  In some cases (2008, 2010, 2013), the results were reworked and submitted for publication in a research journal.}

\underline{Algebra Short Course Instructor}: (2002, 2003, 2014, 2015) \\
{\ss I designed and delivered a course of 12 contact hours, on a topic in Algebra appropriate for junior-level Mathematics students.} 


\large
\section{\bf Departmental Service (major)}
\vspace{3 mm}
\normalsize
\underline{Associate Chair}: Fall 2016 - present. \\ 
{\ss Ongoing duties include designing and maintaining the department teaching schedule, managing student enrollment (ROR), and hearing/ruling upon academic dishonesty cases.  As part of this position, I have also designed and implemented a force add request management system (Spring 2017, prior to ROR), designed a research map for the department web page (Summer 2017), organized a department retreat (Fall 2017), and assembled an archive of course materials for the department (Spring - Summer 2019).}

\underline{Governance Committee}: Fall 2017 - present (Chair), 2008-09, 2006-07. \\
{\ss During the 2017-2018 and 2018-19 academic years, I worked with three other faculty members and in consultation with the department to produce a completely new governance document.  While many policies were borrowed from the old document, the new document is organized very differently, in a form designed for ease of use and updating.  Much work was necessary to eliminate redundancy, update obsolete statements, ensure consistent with university policy, and firm up various definitions.  The new document was adopted by the department in March 2019.}

\underline{Department self-study co-author}: Summer 2016. \\
{\ss I helped Patrick Dowling and Doug Ward write the department self-study document in advance of the program review conducted in October 2016.}

\underline{Chair of Department Tenure Committee}: Fall 2014 - Fall 2017.\\
{\ss My duties were to call and preside or meetings of the committee each Spring to discuss progress and write review letters for probationary faculty members.  During the fall, meetings were held to consider and vote upon applications for tenure and promotion to Associate Professor.}

\underline{Department Graduate Committee}: Fall 2007 - Spring 2014 and Fall 2001 - Spring 2004.\\
{\ss The Graduate Committee considers all issues pertinent to the department's graduate degree programs, including course approvals, curriculum changes, and assistantship offers to program applicants.}

\underline {Department web page}: design and maintenance, Fall 2009 - Spring 2014.\\
{\ss Prior to the uniformization of university web pages implemented in 2014, the department maintained its own website.  I wrote the code for those pages and updated the relevant information each year.}

\underline{Chair of Oxford Math lecturer search committee}: 2013-14.

\underline{Oxford Math tenure-track search committee}: 2016-17, 2014-15, 2011-12, 2002-03.

\underline{Hamilton tenure-track Math search committee}: 2009-10.

\underline{Mathematics Committee}: Chair 2006-08 and Secretary 2002-2006.\\
{\ss This committee, which was in existence from 2002 through 2009 in the (joint) department of Mathematics and Statistics, considered matters specific to mathematics and communicated a recommendation to the department.}

\underline{Chair of {\em ad hoc} Committee for Peer Review of Teaching}: Spring 2005.\\
{\ss This commitee devised a departmental policy for peer review of teaching (for tenure-track faculty), which was then approved by the department and incorporated into the governance document.}


\large
\section{\bf  Departmental Service (minor)}
\vspace{3 mm}
\normalsize

\underline{Department Assessment Report (author)}: 2018. \\
{\ss I wrote the part of the report on the capstone MTH 425, which I had taught that year.}

\underline{Barney Fellowship Selection {\em ad hoc} Committee}: 2016 and 2019. \\
{\ss This committee was convened to consider applications for the M. Pauline Priest Barney Fellowship.}

\underline{Department Retreat}: 2017. \\
{\ss I organized and conducted an on-campus retreat for the Department of Mathematics, in which we discussed various issues of concern, some proceeding from the 2016 program review.}  

\underline{Chair of {\em ad hoc} Online Teaching Evaluation Committee}: Spring 2012 and Fall 2013.\\
{\ss This committee was tasked with drafting or revising department-specific questions on form for student evaluation of teaching, as the university was transitioning from evaluations on paper to online evaluations.}

{Computer Committee}: Fall 2013 - present.

{Departmental Library Liaison}: 2004-06.

{Colloquium Committee}: 2003-04.

{Mathematics Steering Committee}: 2000-2001.

\large
\section{\bf Service to the University}
\vspace{3mm}
\normalsize

Graduate Council (alternate), Fall 2019.

College of Arts and Sciences Committee on Committees, 2016-2018.

University Library Committee, Fall 2010 - Spring 2013.

CAS Committee for the Review of Chairs and Program Directors: Fall 2010 - Spring 2012.

Graduate Council Financial Assistance Subcommmittee: Fall 2010 - Spring 2012.

Harrison Scholarship Screening Committee, 2009.

Honors and Scholars Program Advisory Committee: Fall 2007 - Spring 2010.

Graduate Council Natural Sciences Subcommittee: Fall 2007 - Spring 2008.

\newpage

\large
\section{\bf Student-Centered Service}
\vspace{3mm}
\normalsize

Chapter Advisor to Pi Mu Epsilon: Fall 2005 - Spring 2008 and Fall 2009 - Spring 2010.

Invited oration to Pi Mu Epsilon: Fall 2000, Fall 2004, Fall 2010, Fall 2015.


\large
\section{\bf Service to the State}
\vspace{3mm}
\normalsize


\underline{Open Educational Resource consultant for Ohio Department of Education}: 2021 \\
{\ss This work involves the review of open resource materials for appropriateness with respect to learning outcomes in Linear Algebra and Differential Equations.}

\underline{Ohio Board of Regents Math TAG}: (Panel lead since Summer 2013,  member since Fall 2010) \\
{\ss The Mathematics Transfer Assurance Guide (TAG) Panel evaluates mid-level mathematics courses (Calculus III, Linear Algebra, Differential Equations) at state-funded institutions within Ohio for appropriateness for transfer credit.  The role of panel lead is analogous to that of committee chair.}

\underline{Program Proposal Reviewer}: 2012-2013 \\
{\ss In late 2012, Shawnee State University submitted a pre-proposal for a Master's degree program in Mathematics.  I wrote a review of the pre-proposal and deemed it solid enough to proceed to a full proposal.  The latter was submitted in 2013, and I completed a review of it also.}

\underline{MAGS Thesis Reviewer}: 2010 \\
{\ss I wrote a review of a thesis submitted to the Midwest Association of Graduate Schools (MAGS) for an award.}


\large
\section{\bf  Conference Organization}
\vspace{3mm}
\normalsize

Co-organizer (with Beata Randrianantoanina and Patrick Dowling) of {\em 
Undergraduate Research}: Oxford, OH: September 28-29, 2013.

Co-organizer (with Louis DeBiasio, Tao Jiang, Zevi Miller, and Dan Pritikin) of 
{\em MIGHTY LIV}: Oxford, OH; April 6, 2013.  

Co-organizer (with Paul Larson and Zevi Miller) of {\em The Mathematics of 
Finance}: Oxford, OH; September 30th - October 1st, 2011.

Co-organizer (with Patrick Brosnan and Roy Joshua) of {\em Algebraic
Cycles II: Progress and Prospects}: Columbus, OH: March 24-29, 2008.

Co-organizer (with Paul Larson and Dan Pritikin) of {\em
Recreational Mathematics}: Oxford, OH; September 26-27, 2008.

Co-organizer (with Paul Larson and Bruce Magurn) of {\em Number
Theory} (Miami University Fall Conference): Oxford, OH; September
28-29, 2007.

Co-organizer (with Roy Joshua and Bruce Magurn) of {\em Conference
on Algebraic Cycles}: Oxford, OH; March 5-6, 2003.

Co-organizer (with Linda Eroh and Carmen Schabel) of Project NExT
Special Session {\em Teaching Students to Write Proofs}, MAA
MathFest: Burlington, VT; July 31st, 2002.
	
\newpage	

\large
\section{\bf Other Service to the Profession}
\vspace{3mm}
\normalsize

\underline{Journal referee}: \\
{\ss J. of Pure and Applied Algebra, Communications in Algebra, Hokkaido J. of Mathematics, Clay Mathematics Institute Proceedings, 
Discussiones Mathematicae, Ars Mathematica Contemporanea, American Mathematical Monthly, J. of K-Theory, Semigroup Forum, Involve, Rocky 
Mountain J. of Mathematics, Taiwanese J. of Mathematics, Korean J. of Mathematics, Electronic J. of Combinatorics, J. of Integer Sequences, Pacific J. of Mathematics, Hacettepe J. of Mathematics, Punjab J. of Mathematics.}

Reviewer for Math Reviews (2 reviews).

Judge for Undergraduate Research Poster Session (Joint Mathematics Meetings), January 2002.   


\large
\section{\bf Programming languages}
\vspace{3mm}
\normalsize

Working knowledge of Python, \Cplusplus, Fortran and Matlab.



\large
\section{\bf References}
\vspace{3mm}
\normalsize

%Patrick Dowling, Department of Mathematics, \texttt{dowlinpn@miamioh.edu} \\
%Beata Randrianantoanina, Department of Mathematics, \texttt{randrib@miamioh.edu} \\
%Doug Ward, Department of Mathematics, \texttt{wardde@miamioh.edu}. 

Dennis Burke, Department of Mathematics (emeritus), \texttt{burkedk@miamioh.edu} \\
Louis DeBiasio, Department of Mathematics, \texttt{debiasld@miamioh.edu} \\
Paul Larson, Department of Mathematics, \texttt{larsonpb@miamioh.edu} \\
Doug Ward, Department of Mathematics, \texttt{wardde@miamioh.edu}.




%\section{Awards and Honors}

%\vspace{5 mm}



%\begin{itemize}

%\item
%John Harvard Scholarship, Harvard University, 1992-1995.

%\item
%Phi Beta Kappa, Harvard University, 1995.

%\item
%Brown University Fellowship, 1995-1996.

%\item
%Brown University Teaching Assistantships, various semesters,
%1995-2000.

%\item
%NSERC (Government of Canada) Level A Graduate Fellowship, 1995-1997.

%\item
%NSERC Level B Graduate Fellowship, 1997-1999.

%\item
%Brown University Outstanding Graduate Teaching Award, 2000.

%\item
%Sigma Xi, Brown University, 2000.

%\item
%Sigma Xi Outstanding Research Award, 2000.

%\item
%Nominated for Alumni Distinguished Educator Award (Miami University), 
%2008.

%\end{itemize}

%\large
%\section{\bf Professional Memberships}
%\vspace{3 mm}
%\normalsize

%American Mathematical Society \\
%Society of Catholic Scientists


% I use RCS for revision control.  This part simply puts the revision
% information at the end. Comment this out if you don't use RCS or CVS.
%\begin{center}
%\vspace{\fill}\ \newline
%{\tiny \rm $ $RCSfile: cv-us.tex,v $ $ }
%{\tiny \rm $ $Date: 2008/06/30 $ $ }
%{\tiny \rm $ $Revision: 1.32 $ $ }
%\end{center}

\end{resume}

\end{document}




















\section{\bf Research Experience}

\begin{format}
\title{l}\dates{r}
\employer{l}\location{r}
\body
\end{format}

\title{Research Assistant}
\employer{Pat Bayer and Paul Ellickson}
\location{Duke University}
\dates{2007--2008}
\begin{position}
  Estimation of dynamic discrete choice models and dynamic games in
  continuous time.
\end{position}

\title{Research Assistant}
\employer{Han Hong}
\location{Duke University}
\dates{2006--2007}
\begin{position}
  Estimation and identification of sequential games of complete information.
\end{position}

\title{Research Assistant}
\employer{Jacob\ Vigdor}
\location{Duke University}
\dates{2005--2006}
\begin{position}
  Empirical analysis of the effects of community homogeneity on voting
  behavior due to individual perceptions of relative economic well-being (See
  \href{http://www.nber.org/papers/w12371}{NBER Working Paper 12371}, July
  2006).
\end{position}

\title{Research Assistant}
\employer{Moody\ Chu and Robert\ Funderlic}
\location{N.C. State University}
\dates{2003--2004}
\begin{position}
  Investigation of data mining and clustering methods and their
  applications. Development of updating methods for the centroid
  decomposition.
\end{position}

\title{Research Assistant}
\employer{Willy\ Hereman}
\location{Colorado School of Mines}
\dates{2002}
\begin{position}
  Development of a symbolic software package to compute exact hyperbolic and
  elliptical solutions to systems of nonlinear partial differential
  equations.
\end{position}

%---------------------------------------------------------------------------

\section{\bf Teaching Experience}

\begin{itemize}

\item Teaching Assistant, Econometrics II (Ph.D.), Duke University, Spring 2007
\item Instructor, Microeconomics Qualifier Camp, Summer 2006
\item Teaching Assistant, Game Theory (Ph.D.), Duke University, Spring 2006
\item Teaching Assistant, Microeconomic Analysis (Ph.D.), Duke
  University, Fall 2005
\item Mathematics Tutor, North Carolina State University, 2001
\end{itemize}

%------------------------------------------------------------------------------

%\section{\bf Employment}
%
% \begin{format}
% \employer{l}\location{r}
% \title{l}\dates{r}
% \body
% \end{format}
%
% \title{Computer Technician}
% \employer{Hurley Technology Resources}
% \location{Jefferson, NC}
% \dates{1999--2000}
% \begin{position}
% Windows and UNIX networking for businesses and internet service providers,
% in- and out-of-house PC repair, sales, technical support, and database driven
% web site design. Supervisor: Cyrus Hurley.
% \end{position}

%%===========================================================================%%
\newpage
\opening

\section{\bf Working Papers}

``Learning by doing in DRAM production'' (2007).

``Sequential Monte Carlo Methods for Estimating Dynamic Discrete Choice
Models'' (2007).

``Structural Estimation of Sequential Games of Complete Information'' (2006).

``Updating the Centroid Decomposition with Applications in LSI,''
with M.\ T.\ Chu, D.\ Ca\~{n}as, R.\ E.\ Funderlic, N.\ Orlowski,
and D.\ Schlorff (2004).

``The Effects of Ties on Convergence in K-Modes Variants for Clustering
Categorical Data,''
with N.\ Orlowski, D.\ Schlorff, D.\ Ca\~{n}as, M.\ T.\ Chu,
and R.\ E.\ Funderlic (2004).

%------------------------------------------------------------------------------

\section{\bf Scientific Software}

Blevins, J., J. Heath, and W. Hereman (2002):
\href{http://www.mines.edu/fs_home/whereman/software/PDESolutionTester/}
{\texttt{PDESolutionTester}},
A Mathematica program for the symbolic verification of exact solutions of
nonlinear partial differential equations.

%------------------------------------------------------------------------------

%\section{\bf Presentations}
%
%``Solving Nonlinear Wave Equations with Mathematica,''
%North Carolina State University, April 23, 2003.

%---------------------------------------------------------------------------

\section{\bf Service}

\begin{itemize}
\item Vice President, Economics Graduate Student Council, Duke University,
  2006--2007
\item Vice President, Society of Undergraduate Mathematics, North Carolina
  State University, 2001--2003
\end{itemize}

%%---------------------------------------------------------------------------%%

\section{\bf Honors and Awards}
\begin{itemize}
\item Summer Research Fellowship, Duke University, 2007.
\item Institute on Computational Economics Fellow, University of Chicago, 2006.
\item Outstanding Teaching Assistant Award, Duke University, 2006
\item Undergraduate Research Award, North Carolina State University, 2004
\item \href{http://www.pbk.org/}{Phi Beta Kappa}, 2003
\end{itemize}

%%---------------------------------------------------------------------------%%
\section{\bf Computer Skills}

\begin{itemize}
\item Expert: \Cplusplus, Fortran, Stata, Perl, \LaTeX, Linux
\item Intermediate: Matlab, SQL, Ruby, SAS, Mathematica
\item Basic: Assembly, Java
\end{itemize}

%%---------------------------------------------------------------------------%%



%%===========================================================================%%
